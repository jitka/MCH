% 5. Formal Results
%  - Here, we state and prove the following formal properties of our proposed solution. 


\section{Formal Results}
\begin{lemma}
%[Lemma]
Set of edges found by query is Pareto set of shortest paths.
\end{lemma}
\begin{proof}
\begin{itemize}
\item[$\subseteq$] 

Let $p_{s,t}$ be arbitrary path from Pareto set $P(s,t)$. 

Suppose that $t$ is the most important node in the path $p_{s,t}$, 
then path $p_{s,t}$ should be found by forward search
and together with path $p_{t,t}$, $l(p_{t,t})=0$ 
founded by backward search its represents original path.
If $p_{s,t}$ is the single edge, then forward search founds it.
If $p$ is the second most important node, then we denote $v$ the
third most important node. 
By the time when $w$ was contracting, all other nodes had been contracted
and $s,t$ was adjacent to $v$. So the shortcut $(s,t)_j$ was created, and forward search uses it.
If $v \ne s$ is the second most important node on the path,
then the sub-path $p_{s,v}$ is found by the forward search by induction and
then sub-path $p_{v,t}$ is also found by the forward search.

The case when $s$ is the most important node on path $p_{s,t}$ is very similar. 

Let $v$ be the most important node on path $p_{s,t}$. 
Sub-path $p_{s,v} \in \Pi(s,v)$ 
if not than the $p' \in \Pi(s,v)$ which dominates $p_{s,v}$ could be substituted
to $p_{s,t}$ and the resulting paths $p'_{s,t}$ dominates $p_{s,t}$ which contradict
$p \in \Pi(s,t)$.
Similarly, $p_{v,t} \in \Pi(v,t) $. Path $p_{s,v}$ is found by forward search end path
$p_{v,t}$ by backward search.

\item[$\supseteq$] 
Each path found by multi-weighted CH corresponds to the path in the original graph. 

\end{itemize}
\end{proof}

\subsection{Highway dimension}

% TODO (origin-destinetion) 

Parameter Highway Dimension was introduced in \cite{abraham2010highway}.
There are few slightly different definition. 
We use definition based on (\cite{abraham2016highway}, Lemma 4.3)
which  use \emph{ Sparse Shortest Path Hitting Set (SPHS)}.
This definition could be easily transformed for multi-weighted directed
multi-graphs. We changed definition of ball and multiscale SHPS.


\begin{definition}
Given a non-negative vector $r \in \mathbb{R}^+$, let 
$B_{u,r} = \{v \in V, |P(u, v)| \le r \}$ 
be the ball of radius $r$ centered at $u$.
\end{definition}

\begin{definition}
Given a set of paths $P$, we say that $H \subseteq V$ is a hitting set for $P$ if every
path in $P$ contains a vertex in $H$. 
\end{definition}

\begin{definition}
For $r > 0$, an $(h, r)-$SPHS is a hitting set $C \subseteq V$ such that
$\forall v \in V, |B_{2r} (v) \cap C| \le h$.
\end{definition}

\begin{definition}
Let $D$ be the diameter and $k = \log(\max(D))$ logarithm of value of its biggest coordinates. 
$D_k = D$, when $D_i$ is vector $(w_0,w_1,\dots)$ then $D_{i-1} = (w_0/2,w_1/2,\dots)$. 
A \emph{multiscale SPHS} of $(G,w)$ is a collection of sets $C_{D_i}$ for $0 \le i \le k$.
\end{definition}

\begin{definition}
If the highway dimension of $(G,w)$ is h, then there exists a multiscale
SPHS such that $|C_{\log D}| \le h$ and $|C_i| \le h|C_{i+1}|$ for $0 \le i < \log D$.
\end{definition}

In order to get time and space estimates of MCHp and MCHq we need to count shortcuts in
the graph. Let $Q_{D_i} = C_{D_i} \setminus \Cup^{k}_{j=i+1}C_{D_j}$.

\begin{lemma}
For fixed $v$ and $j$, the number of shortcuts $(v,w)$ with $w\in Q_j$ is at most $h$.
\end{lemma}

\begin{theorem}
\label{shrotcutsNumber}
If $(G,w)$ has highway dimension h, then preprocessing based on
multiscale SPHS produces a set of shortcuts $E+$ such that degree of every vertex in
$G(V, E \cup E+)$ is at most $h + h \log D$ and $|E+| \le nh\log D$.
\end{theorem}

The proof is exactly same as in (\cite{abraham2016highway}, Lemma 6.1, Theorem 6.2)

Time complexity of MCHp 

Time complexity of MCHq. By theorem \ref{shrotcutsNumber} we know that degree is 
bounded by $h \log D$. When the Meta Search Algorithm use Fibonacci heaps  


% During query faze we can use bidirectional search 
% \cite{ma2013using}
% with heuristic. 


% There is an ordering of vertices
% such that CH query with range optimization takes
% O((∆ + h log D)(h log D)) time, and a poly-time computable
% ordering such that the query takes O((∆ +
% h log n log D)(h log n log D)) time.

% Theorem 5.3. Consider the variant of CH with additional
% shortcuts. The total number of shortcut
% edges is bounded by |E∗
% | = O(nh log D) or
% O(nh log n log D) for poly-time preprocessing. The
% query takes O((∆ + h log D)(h log D)) time or O((∆ +
% h log n log D)(h log n log D)) for polynomial-time preprocessing.
