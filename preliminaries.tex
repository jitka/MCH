% 2. Preliminaries
%  - First, let's define MOSP (graph, solutions, etc.).
%  - We're going to be using MLS algorithm in our solution. 
%    MLS alg. is ... (explain it here instead of its own section)

\section{Preliminaries}
\label{secPreliminaries}

First of all, let us describe the Multi-Objective Shortest Path Problem (MOSP) and the Multicriteria Label-Setting (MLS) algorithm, which is relevant to our own solution.

Let $G(V,E)$ be a directed multi-graph where $V$ is a set of nodes and $E$ a set
of edges with $|V| = n$ and $|E| = m$. Since $G$ is a multi-graph, multi-edges
$((u,v)_0,(u,v)_1,\dots)$ are allowed in $E$. For a fixed node $v$, any edge $(u,v)_i$ is called an in-edge and any edge $(v,w)_j$ is considered an out-edge.

Every node is assigned a \emph{weight vector} given by the weight function $w: E \rightarrow \mathbb{R+}^d$. Dimension $d$ represents a number of criteria in a multi-objective search. 
In our experiments, we consider $d=2$ and $d=3$.
We say that a weight vector $a$ is dominated by vector $b$ ($a,b \in \mathbb{R+}^d$) if
$b$ has at least one weight lower than $a$ and no weight higher than $a$. 

Any sequence of edges $(s=u_0,u_1)_*,(u_1,u_k)_*,\dots(u_{k-1},t=u_k)$ is called
\emph{path} $p_{s,t}$. Length of a path is defined as a sum of the weights 
$w(p_{s,t}) = \Sigma_{i=0}^{i<k} (u_i,u_{i+1})$. The set of all non-dominated paths between $s$ and $t$ is called {\em Pareto set}
$P(s,t)$. Given a pair of nodes $s,t$, the aim of MOSP is to find a Pareto set $P(s,t)$.

\vskip 5mm

%\subsection{Meta Search Algorithm}
Many typical search algorithms in this context are related either to original Dijkstra algorithm~\cite{dijkstra1959note} or to Multicriteria Label-Setting (MLS) algorithm~\cite{martins1984multicriteria}. Let us now present its generalization in order to establish the notation.

MLS algorithm (see Pseudocode \ref{MLSalg}) is designed to search for paths in multi-weighted directed multi-graphs. It can answer three types of queries:
\begin{itemize}
\item Find all non-dominated shortest paths from source node $s$ to target node $t$ ($s$ is sometimes called {\em origin} and $t$ is called {\em destination}).
    % TODO přepsat s t na o d (origin destination)
\item Find all non-dominated shortest paths from $s$ to every $t \in T \subseteq V$.
\item Find all non-dominated shortest paths from $s$ to every other vertex in the graph ($\forall t \in V$).
\end{itemize}

Key part of the algorithm is assigning so-called \emph{labels} $l\in \mathcal{R}^d$ to graph nodes.
Each label $l$ of a node $v$ contains the length $w(l)$  of one possible path from $s$ to $v$. Given a node pair $s,t$ and a label $l$, we denote the name of its corresponding node as $node(l) = v$ and the name of previous node on the path from $s$ to $v$ as $prev(l)$.
Each node has a set of non-dominated labels called \emph{label bag} $B(v)$.

As the first step of the algorithm, we assign labels in the following manner: $B(s)=0^d$ and $\forall v \ne s: B(v)=\emptyset$.
At every subsequent step of the algorithm, one label is \emph{expanded}. 
During the expansion, all the outgoing edges are considered -- the outgoing edge, together with the previous path to the label make up the path that is added to corresponding label bag. Then, we check whether there is a label which dominates another.
\todo[inline]{nie som si isty, ci su predchadzajuce 2 vety spravne, plus nie su uplne jednoducho zrozumitelne. skontroluj to prosim a ked sa da, vylepsi tu formulaciu tak, aby bola jasnejsia}

Data structure {\em open\_labels} is typically used when implementing MLS. It is a priority queue with all the labels which were created, but not yet expanded. The priority function needs to guarantee that top label is never dominated by any other label in that queue.

The MLS algorithm stops when every label is expanded. At that time, the labels in bags $B(t), t \in T$ represent the lengths of the paretoset of shortest path.\todo{Veta zacinajuca na ``At that time\ldots'' asi nie je formalne uplne ok.} Resulting path can be reconstructed recursively using $prev(l)$.

%\todo[inline]{je třeba dokazovat korektnost nebo je to zjevne ci zname?} -> netreba

\renewcommand{\lstlistingname}{Pseudocode} 
\begin{lstlisting}[caption={MLS Algorithm},label=MLSalg,captionpos=t,float,abovecaptionskip=-\medskipamount]
Input: source node s, set of target nodes T
for all nodes v:
	B(v) <- \emptyset
B(s) <- 0^d
B(t) <- \infinity^d for all t \in T
open_labels <- s
while open_labels is not empty:
	l <- some label from open_labels
    if w(l) is dominated by all labels in B(t) \in T
    	continue
    for every outgoing edge e=(node(l),w)_i
    	for every label k from bag B(w):
        	if k is dominated by l+w(e):
            	remove k from bag B(w)
        if l+w(e) is not dominated by any label from B(w)
        	add new label with weight (l+w(e)) to B(w)
Output: B(t) for every t \in T
\end{lstlisting}

%\todo[inline]{sazba pseudokodu}
