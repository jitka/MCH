%  3.1. Stage 1: Preprocessing
%    - Explain MCHp in natural language here.
%      Get inspiration from what you wrote on whiteboard yesterday. 
%      Feel free to use some numbered lists to make it easy to read and understand.
%      If needed, make and use some graph pictures.
%    - Show the MCHp pseudocode.


\subsection{Stage 1: Preprocessing}
\label{subsecStage1}

Preprocessing contract all nodes one by one from least important to the most important. 
Cost vector of all Pareto optimal paths between two nodes has to last after contraction.

In our implementation the importance of node is measured by heuristic function $f$
described in section \todo{reference to experimens subsection}. 
First we initialize $f$ values for every nodes. In each step we choose node $v$ with 
lowest $f$ and update $f$ of all nodes adjacent to $v$.
In formal results the importance is based on highway dimension.

At $i$-th step least important node $v$ got $rank(v) = i$ and is \emph{contracted}. 
Contraction of node $v$ consists removing and archiving all edges adjacent $v$, removing 
and archiving node $v$ and adding shortcuts. 

We need to keep distance of all Pareto optimal paths in the graph. 
For each two edges $(u_i,v)_k,(v,w_j)_l$ we need to know if there
are part of a Pareto optimal path. If we found path $p(u_i,w_j)_m$ which dominates
$w((u_i,v)_k)+w((v,w_j)_l)$ then the combination is not needed. 
We call $p(u_i,w_j)_m$ a \emph{witness} for these two edges. 

To found shortcuts, we run witness search and add shortcut for each pair of edges
which do not have the witness. Shortcut is new edges with weight equal to sum
of weights of edges and with note, which node was used to create it.

Note that shortcut could be added even if it has witness. 
Unnecessary shortcut do not violate correctness, but it could make preprocessed graph larger.

By this process multiple edges with different weights could be created. 

When all node was removed all nodes and edges of preprocessed graph are in the archive.

See pseudocode in \ref{MCHp}

\subsubsection{Witness search}

Witness search is used to found which shortcut we need to add when we contract 
node $v$. We denote in-edges $(u_i,v)_k$ and out-edges $(v,w_i)_l$.
\todo{mám obrázek zatím na papiře jak může vypadat okoli v, pošlu ho do chatu, jestli ho ma smysl přepisovat}

For each in-edge $(u_i,v)_k$ we run separate search which founds all shortcuts from $u_i$.
We use an algorithm based on MLS with query $u_i-\{w_0,\dots w_k\}$. 

Structure open\_labels is implemented as a priority queue with alphabetical ordering. 
Top of quote contain labels with the lowest first weight which are ordered by second weight and so on. 
In such ordering, each label comes after all open labels which dominate it.

Algorithm MLS do not expand label when it is dominated by at least one path to every goal node $w_j$. 
It stops when there are no open labels and is goal nodes are all witness path
or it stops or after some time limit founding only subset of witness path. 

\begin{lstlisting}[caption={MCHp},label=MCHp,captionpos=t,float,abovecaptionskip=-\medskipamount]
Input: Graph G
for i from 0 to n-1
    v <- least important node
    rank(v) = i
    witnesses <- witness search
    for every pair of edges (u,v)_i,(v,w)_j:
    if p(u,v,w) is not dominated by any witness:
        add shortcut (u,w)_0
            w((u,w)_0) = w((u,v)_i)+w((v,w)_j)
            note((u,w)) = v
    archive all out-edges (v,w) 
    archive all in-edges (u,v)
    archive v
Output: Archived nodes, edges, rank of nodes.
\end{lstlisting}
