%  3.1. Stage 1: Preprocessing
%    - Explain MCHp in natural language here.
%      Get inspiration from what you wrote on whiteboard yesterday. 
%      Feel free to use some numbered lists to make it easy to read and understand.
%      If needed, make and use some graph pictures.
%    - Show the MCHp pseudocode.


\subsection{Stage 1: Preprocessing}
\label{subsecStage1}

Preprocessing algorithm (see Pseudocode \ref{MCHp} below) contracts all the graph nodes one by one in the order of their importance (starting with the least important), while keeping the cost vector of all the Pareto optimal paths unchanged. Importance of a node is expressed by a heuristic function $f$. In our implementation, we used a highway dimension (described in section \ref{secFormalResults}).
 
First, we initialize the $f$ values for all the nodes. Then, at each step of the algorithm, we select the node $v$ with lowest value of $f$ and update $f$ of all the nodes adjacent to $v$. 
At $i$-th step, the least important node $v$ is assigned a $rank(v) = i$ and is \emph{contracted}. 
Contraction of a node $v$ consists of removing and archiving all the edges adjacent to $v$, removing and archiving the node $v$ itself and adding {\em shortcuts}. 

We need to keep the distance of all Pareto optimal paths in the graph unchanged. For every two edges $(u_i,v)_k,(v,w_j)_l$, we need to know if they are part of a Pareto optimal path. If we found a path $p(u_i,w_j)_m$ which dominates $w((u_i,v)_k)+w((v,w_j)_l)$, then the combination is not needed. We call $p(u_i,w_j)_m$ a \emph{witness} for these two edges. 

To find the {\em shortcuts}, we run so-called {\em witness search} (described below) and add a shortcut for each pair of edges which do not have a witness. A shortcut is a new edge with the weight equal to the sum of weights of edges\todo{sum of weights of which edges?}. We also keep a note about which node was used to create which shortcut.

Note that a shortcut could be added even if a witness exists for it. Unnecessary shortcuts do not violate the correctness. They can only make the preprocessed graph larger.

%By this process, multiple edges with different weights could be created. 

Finally, when all the nodes are removed in this manner, we have the preprocessed graph in the archive including all its nodes and edges.

\paragraph*{Witness Search}
The Witness Search is used to determine which shortcuts we need to add when we contracting a node $v$. 

First, we denote all the in-edges as $(u_i,v)_k$ and out-edges as $(v,w_i)_l$. For each in-edge $(u_i,v)_k$, we run a separate search which finds all the shortcuts from $u_i$. For that, we use a MLS algorithm with query $u_i-\{w_0,\dots w_k\}$. 

Structure open\_labels is implemented as a priority queue with alphabetical ordering. 
Top of quote contain labels with the lowest first weight which are ordered by second weight and so on. In such ordering, each label comes after all open labels which dominate it.\todo[inline]{Cely tento odstavec nechapem. Na co je open\_labels? Aky quote? Tuto cast potrebujeme lepsie vysvetlit. Doplnujuca otazka: nie su to nahodou len nejake zbytocne implementacne detaily, ktore bude lepsie vynechat?}

The MLS algorithm does not expand a label when it is dominated by at least one path to every goal node $w_j$. It stops either when there are no open labels and its goal nodes are all witnesses,
or when some time limit was reached (finding only a subset of witnesses). 
\todo[inline]{nemame na witness search pseudokod?}

\renewcommand{\lstlistingname}{Pseudocode}
\begin{lstlisting}[caption={MCHp},label=MCHp,captionpos=t,float,abovecaptionskip=-\medskipamount]
Input: Graph G
for i from 0 to n-1
    v <- least important node
    rank(v) = i
    witnesses <- witness search
    for every pair of edges (u,v)_i,(v,w)_j:
    if p(u,v,w) is not dominated by any witness:
        add shortcut (u,w)_0
            w((u,w)_0) = w((u,v)_i)+w((v,w)_j)
            note((u,w)) = v
    archive all out-edges (v,w) 
    archive all in-edges (u,v)
    archive v
Output: Archived nodes, edges, rank of nodes.
\end{lstlisting}
