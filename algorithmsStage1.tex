%  3.1. Stage 1: Preprocessing - Explain MCHp in natural language here.  Get
%  inspiration from what you wrote on whiteboard yesterday.  Feel free to use
%  some numbered lists to make it easy to read and understand.  If needed, make
%  and use some graph pictures.  - Show the MCHp pseudocode.


\subsection{Stage 1: Preprocessing} \label{subsecStage1}

Preprocessing algorithm (see Algorithm \ref{MCHp} below) contracts all the
graph nodes one by one in the order of their rank (starting with the lowest
rank), while keeping the cost vector of all the Pareto optimal paths unchanged.
Rank of a node is assigned by a heuristic function $f$. In our implementation,
we used a function described in section \ref{secExperiments}.
%Higway dimension in theretical resutls

\vskip 5mm
\begin{algorithm}[H]
    \SetAlgoLined
    \LinesNumbered
    \SetKwInOut{Input}{input}
    \SetKwInOut{Output}{output}
    \SetKwComment{Comment}{$\triangleright$\ }{}
    \SetKw{Continue}{continue}
    \caption{MCHp algorithm for graph preprocessing}
    \label{MCHp}
    \Input{ Graph $G$ with weight function $w: E \rightarrow R_+^d$. }
    \For{\upshape $i$ from $0$ to $n-1$} {
        $v \leftarrow $ node with the minimal rank
        $rank(v) = i$ \\
        witnesses $\leftarrow$ witness search \\
        $n \leftarrow 0$ \Comment*[r]{Number of new shortcuts} 
        \ForEach{\upshape pair of edges $(u,v)_i,(v,w)_j$ } {
            \If{\upshape $p(u,v,w)$ is not dominated by any witness } {
                add shortcut $(u,w)_n$ \\
                $w((u,w)_n) = w((u,v)_i)+w((v,w)_j)$ \\
                $contracted\_node((u,w)_n) = v$ \\
                $n = n+1$ \\
            }
        }
        remove all out-edges $(v,w)$ and in-edges$(u,v)$ from $G$ \\
        remove $v$ from $G$ \\
        put $v$ and adjacent edges into archive 
    }
    \Output{ Archived nodes, edges, rank of nodes. }
    \vskip 5mm
\end{algorithm}
\vskip 5mm

First, we initialize the $f$ values for all the nodes. Then, at each step of
the algorithm, we select the node $v$ with lowest value of $f$ and update $f$
of all the nodes adjacent to $v$.  At $i$-th step, the lowest-ranked node $v$
is assigned a $rank(v) = i$ and is \emph{contracted}.  Contraction of a node
$v$ consists of removing and archiving all the edges adjacent to $v$, removing
and archiving the node $v$ itself and adding {\em shortcuts}. 

\vskip 5mm
We need to keep the distance of all Pareto optimal paths in the graph
unchanged. For every two edges $(u_i,v)_k,(v,w_j)_l$, we need to know if they
are part of a Pareto optimal path. If we find a path $p_{u_i,w_j}$ which
dominates $w((u_i,v)_k)+w((v,w_j)_l)$ , then the combination is not needed. We
call such $p_{u_i,w_j}$ a \emph{witness} of edges $(u_i,v)_k,(v,w_j)_l$. 

To find the {\em shortcuts}, we run so-called {\em witness search} (described
below) and add a shortcut for each pair of edges which do not have a witness.
For a pair of nodes $u_i, w_j$ there is $m$ edges $(u_i,v)_k,(v,w_j)_l$
with no witness. Shortcuts are new edges
$(u_i,w_j)_0, (u_i,w_j)_1 \dots (u_i,w_j)_m$ 
%\todo[inline]{ta nula tam je trošku haluzná protože ve stejném kroku můžu
%přidat i druhou zkratku (ui,wj)1, netusim jak to napsat aby to nebylo matoci}
with weights $w((u_i,w_j)_*) = w((u_i,v)_k)+w((v,w_j)_l)$.  We also keep a note
about which node was used to create which shortcut.  Note that a shortcut could
be added even if a witness exists for it. Unnecessary shortcuts do not violate
the correctness. They can only make the preprocessed graph larger (in some
cases, adding unnecessary shortcuts can even speed up the preprocessing). 

%By this process, multiple edges with different weights could be created. 

Finally, when all the nodes are removed in this manner, we have the
preprocessed graph in the archive including all its nodes and edges.

\paragraph*{Witness Search} The Witness Search is used to determine which
shortcuts we need to add when we contracting a node $v$. 

First, we denote all the in-edges as $(u_i,v)_k$ and out-edges as $(v,w_i)_l$.
For each node $u_i$ connected with $b$ by in-edge $(u_i,v)_*$, we run a
separate search which finds all the shortcuts from $u_i$. For that, we use a
MLS algorithm with query from $u_i$ to $\{w_0,\dots w_k\}$. 

The MLS algorithm does not expand a label when it is dominated by at least one
path to every goal node $w_j$. It stops either when there are no open labels
and its goal labels represents all witness paths, or when some time limit was reached
(finding only a subset of witnesses). 
%\todo[inline]{nemame na witness search pseudokod?} \todo[inline]{ne mame
%pseudokod na MLS který se od withnes searche lisi jen malinko, stejne tak se
%od nej forward a reverse search lisi jen trochu}


