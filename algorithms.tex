% 3. Proposed Algorithms: Multi-weighted Contraction Hierarchies
%  - Proposed solution consists of two algorithms for two 
%    separate stages: graph preprocessing and the search query  

\section{Proposed Algorithms: Multi-weighted Contraction Hierarchies}
\label{secAlgorithms}

To address the problem of finding the Pareto set of optimal paths in multi-weighted directed multi-graphs, we propose Multi-weighted Contraction Hierarchies (MCH). 
MCH approach consists of two separate stages, each solved by a separate algorithm: contracted graph construction (MCHp) and query (MCHq). 
Both algorithms are generalisation of Contraction Hierachies for single-weighted undirected graphs \cite{geisberger2008contraction}.

\begin{itemize}
  \item {\em MCHp: Preprocessing algorithm} takes a multi-weighted directed multigraph on the input and returns an extended graph with the same set of nodes, but with additional edges. It contracts all the nodes one by one, adding new edges called \emph{shortcuts}, while preserving all the Pareto optimal shortest path lengths. 
Nodes will have a rank ($rank(v) \in \{0, \dots n-1\}$) corresponding to their order of contraction.
The resulting preprocessed graph has all the original nodes $V$, edges $E$ and new set of shortcuts $E_+$.
Each shortcut $(u,v)_i$ corresponds to a path in the original graph $p_{u,v}$, while $w((u,v)_i) = w(p_{u,v})$ and it keeps a note which is later used to reconstruct the original path.
  \item {\em MCHq: Query algorithm} uses the preprocessed graph to find the Pareto set of all
the shortest paths between two nodes $s$ and $t$. MCHq runs a search algorithm in both directions -- from $s$ to $t$ and from $t$ to $s$. After that, it joins the both partial paths into the path in the preprocessed graph and finally reconstructs the path in the original graph. 
\end{itemize}

%  3.1. Stage 1: Preprocessing
%    - Explain MCHp in natural language here.
%      Get inspiration from what you wrote on whiteboard yesterday. 
%      Feel free to use some numbered lists to make it easy to read and understand.
%      If needed, make and use some graph pictures.
%    - Show the MCHp pseudocode.


\subsection{Stage 1: Preprocessing}
\label{subsecStage1}

Preprocessing algorithm (see Pseudocode \ref{MCHp} below) contracts all the graph nodes one by one in the order of their rank (starting with the lowest rank), while keeping the cost vector of all the Pareto optimal paths unchanged. Rank of a node is assigned by a heuristic function $f$. In our implementation, we used a highway dimension (described in section \ref{secExperiments}).

\renewcommand{\lstlistingname}{Pseudocode}
\begin{lstlisting}[caption={MCHp},label=MCHp,captionpos=t,float,abovecaptionskip=-\medskipamount]
Input: Graph G
for i from 0 to n-1
    v <- node with minimal rank
    rank(v) = i
    witnesses <- witness search
    for every pair of edges (u,v)_i,(v,w)_j:
    if p(u,v,w) is not dominated by any witness:
        add shortcut (u,w)_0
            w((u,w)_0) = w((u,v)_i)+w((v,w)_j)
            note((u,w)) = v
    archive all out-edges (v,w) 
    archive all in-edges (u,v)
    archive v
Output: Archived nodes, edges, rank of nodes.
\end{lstlisting}

First, we initialize the $f$ values for all the nodes. Then, at each step of the algorithm, we select the node $v$ with lowest value of $f$ and update $f$ of all the nodes adjacent to $v$. 
At $i$-th step, the lowest-ranked node $v$ is assigned a $rank(v) = i$ and is \emph{contracted}. 
Contraction of a node $v$ consists of removing and archiving all the edges adjacent to $v$, removing and archiving the node $v$ itself and adding {\em shortcuts}. 

We need to keep the distance of all Pareto optimal paths in the graph unchanged. For every two edges $(u_i,v)_k,(v,w_j)_l$ , we need to know if they are part of a Pareto optimal path. If we found a path $p(u_i,w_j)_m$ which dominates $w((u_i,v)_k)+w((v,w_j)_l)$ , then the combination is not needed. We call $p(u_i,w_j)_m$ a \emph{witness} for these two edges. 

To find the {\em shortcuts}, we run so-called {\em witness search} (described
below) and add a shortcut for each pair of edges which do not have a witness.
For pair of edges $(u_i,v)_k,(v,w_j)_l$, shortcuts are a new edges $(u_i,w_j)_0, (u_i,w_j)_1 \dots$ 
%\todo[inline]{ta nula tam je trošku haluzná protože ve stejném kroku můžu přidat i druhou zkratku (ui,wj)1, netusim jak to napsat aby to nebylo matoci}
with weights $w((u_i,w_j)_0) = w((u_i,v)_k)+w((v,w_j)_l)$. 
We also keep a note about which node was used to create which
shortcut.
Note that a shortcut could be added even if a witness exists for it. Unnecessary shortcuts do not violate the correctness. They can only make the preprocessed graph larger.

%By this process, multiple edges with different weights could be created. 

Finally, when all the nodes are removed in this manner, we have the preprocessed graph in the archive including all its nodes and edges.

\paragraph*{Witness Search}
The Witness Search is used to determine which shortcuts we need to add when we contracting a node $v$. 

First, we denote all the in-edges as $(u_i,v)_k$ and out-edges as $(v,w_i)_l$. For each in-edge $(u_i,v)_k$, we run a separate search which finds all the shortcuts from $u_i$. For that, we use a MLS algorithm with query $u_i-\{w_0,\dots w_k\}$. 

The MLS algorithm does not expand a label when it is dominated by at least one path to every goal node $w_j$. It stops either when there are no open labels and its goal nodes are all witnesses,
or when some time limit was reached (finding only a subset of witnesses). 
%\todo[inline]{nemame na witness search pseudokod?}
%\todo[inline]{ne mame pseudokod na MLS který se od withnes searche lisi jen malinko, stejne tak se od nej forward a reverse search lisi jen trochu}



%  3.2. Stage 2: Search query
%    - Explain MCHq in natural language here.
%      Get inspiration from what you wrote on whiteboard yesterday. 
%      Feel free to use some numbered lists to make it easy to read and understand.
%      If needed, make and use some graph pictures.
%    - Show the MCHq pseudocode.

\subsection{Stage 2: Search Query}
\label{subsecStage2}

Multi-weighted Contraction Hierarchies query (MCHq) is based on a bi-directional search.
Each $s$-$t$ search query runs two MLS-like searches on a slightly modified preprocessed graph
and then we combine their results. 
%We do not transform the whole graph in advance -- we merely modify it during queries.

\paragraph*{Forward Search}
Forward search runs on a preprocessed sub-graph with only those oriented edges going from nodes with lower rank to nodes with higher rank (other edges are ignored). It is not necessary to transform the whole graph in advance -- we do it on the go. 
%In the pseudocode of MLS we change line \todo{jina sazba pseudokodu} to \ref{}.
The MLS algorithm is used to search for the shortest path from $s$ to $t$ and obtain the label bags $B(v)$ for all the nodes.

\begin{lstlisting}[caption={Forward Search},label=list:8-6,captionpos=t,float,abovecaptionskip=-\medskipamount]
    for every outgoing edge e=(node(l),w):
    	if rank(node(l)) < rank(w):
        	continue;
\end{lstlisting}

\paragraph*{Reverse Search}
%\todo[inline]{prepisu po schvaleni Forward Searche}
Reverse search is similar to Forward search, but it originates in the node $t$ and runs on a graph with all the edges reversed (implicitly -- we do not need to actually modify the graph structure for this). Again, we only consider those edges, which (after being reversed) go from lower-ranked nodes to higher-ranked nodes. 
%This could by done by replacing line \todo{budu predelavat sazbu pseudokodu aby to slo} by \ref{}
In this case, the MLS search algorithm is used to search for the shortest path from $t$ to $s$ and obtain the label bags $B(v)$ for all the nodes.

\begin{lstlisting}[caption={Reverse Search},label=list:8-6,captionpos=t,float,abovecaptionskip=-\medskipamount]
    for every outgoing edge e=(w, node(l)):
    	if rank(w) < rank(node(l)):
        	continue;
\end{lstlisting}

\todo[inline]{prosim ta skontroluj, ci som tie forward/reverse search odstavce napisal spravne. ak ano, tak len vymaz tuto poznamku}

To find the paths in preprocessed graph, we combine the labels from forward and reverse search. 
For all the nodes $v$ with $B(v) \ne \emptyset$, we join all the combinations of $s$-$v$ and $v$-$t$ paths. 
All non-dominated paths make up the Pareto set of shortest $s$-$t$ paths in the preprocessed graph.
Pareto set of shortest paths in the original graph can be reconstructed recursively using the notes on the shortcuts.



