% 4. Experiments
%  - We've tried both algorithms (MCHp and MCHq) on the New York City road 
%    map benchmark from ... (what was that challenge? include the lint to it)
%  - The full graph had 270K nodes. Benchmark also contained subgraphs with 10,20,30-50K nodes. 
%    There were X queries.
%  - We compared our algorithms to alternatives A, B, C... (reference them - was it only Dijkstra?)
%  - (optional) Link to our implementation is ... 
%  - The experiments were conducted on a machine ... (specifications)
%  - The experiment results are in ... (results table here)
%  - In the table, we see that our algorithms performed better with respect to 
%    number of expanded nodes and with respect to query time (in all cases?).
%    On average, we expanded only X% of the nodes expanded by Dijkstra and our queries 
%    took only Y% of Dijkstra time.
% politika zabijeni whitnessearche
% heuristika


\section{Experiments}
\label{secExperiments}

We experimented with our implementation of MCHp and MCHq algorithms using the New York City road map benchmark from 9th DIMACS Implementation Challenge: Shortest Path\footnote{
\url{http://www.dis.uniroma1.it/challenge9/}} and compared them to MLS algorithm. We were interested in two metrics: number of expanded nodes and computation time.

Graph representing New York City road network consists of 270,000 nodes. In addition to full graph, the benchmark also features smaller subgraphs with 10K, 20K, 30K, 40K and 50K nodes. 

The benchmark problem was defined as a bi-criterial shortest path search. The two objectives were: geographic distance and travel time. 
We also generated additional graphs with a randomized third criterion. Where the original weight vector was $(w_1,w_2)$, we set the new vector to be $(w_1,w_2,rand(0.75,1.5) \times w_1 + rand(0.75,1.5) \times w_2)$ and $(w_1,w_2,rand(0.5,2) \times w_1 + rand(0.5,2) \times w_2)$. 

%\todo[inline]{Tretie (nahodne) kriterium nemusime myslim vobec spominat. Ja by som tento (modry) odstavec proste zmazal/zakomentoval. Ale je to na tebe.}
%\todo[inline]{Me to prislo velmi prekvapive ze to dost dobre funguje i s nekorelovanou souradnici. Nechala bych to tam.}

We computed 1,000 random queries on the subgraphs and 10 queries on the full graph and tri-criteria graphs and averaged the results (times and numbers of extended nodes) among those queries. 
The queries on the full graph had to be hand-crafted in such a way, that would allow the MLS algorithm to finish in a reasonable time (MLS failed to finish within one day using random full graph queries).   
The {\em node rank} heuristic was computed as {\tt shortcut*3 + deleted*3 -~inEdges -~outEdges} in our MCH implementation. The Witness Search time limit was set to 0.25 sec. 

%We compared our algorithms to MLS.
%\todo{( was it only Dijkstra?), psala jsem i namou, ale vycházela mi jen asi o 30\% rychlejsi nez MLS, a nejsem si jista jestli je to ok a nebo mam spatnou imlementaci nebo spatnou heuristiku}

All the algorithms were implemented in Java 1.8. The experiments were conducted on a machine with 
32GB RAM and Intel(R) Core(TM) i7-4770S CPU @ 3.10GHz processor. The experiment results are in Table \ref{result}. 

\begin{table}[h]
\tiny
\caption{Comparison of MLS and MCL}
\label{result}
\begin{tabular}{l|c|cc|ccccc}
 & & \multicolumn{2}{c}{MLS} & \multicolumn{5}{c}{MCH} \\ \cline{3-6} \hline
instance & nodes & query & query & prep & query & query & speed up & speed up \\ \hline
& & (expanded) & (ms) & (sec) & (expanded) & (ms) & expanded & time \\ \hline
full & 270 000 & 9883105 & 494352 & & 36709 & 9289 & 269.23 & 53.22  \\ \hline
11k-10k & 10 000 & 26051 & 32 & 4.3 & 4227 & 56 & 6.16 & 0.57 \\ \hline
2455-20k & 20 000 & 178472 & 607 & 31 & 2055 & 34 & 86.85 & 17.85 \\ \hline
5555-20k & 20 000 & 174808 & 514 & 23 & 2004 & 20 & 87.23 & 25.7 \\ \hline
8765-20k & 20 000 & 53238 & 60 & 12 & 1365 & 5 & 39.00 & 12.0 \\ \hline
110k-30k & 30 000 & 146709 & 278 & 41 & 6853 & 149 & 21.41 & 1.86 \\ \hline
150k-40k & 40 000 & 912326 & 17455 & 1978 & 10307 & 353 & 88.51& 49.45 \\ \hline
200k-50k & 50 000 & 996507 & 16181 & 2142 & 13681 & 711 & 72.84& 22.76 \\ \hline
10k(0.75-1.5) & 10 000  & 38997 & 126 & 17 & 8830 & 643 & 4.42 & 0.20 \\ \hline
10k(0.5-2) & 10 000     & 377541 & 20798 & 734 & 12261 & 353 & 30.80 & 58.92 \\ \hline
20k(0.75-1.5) & 20 000 & 355497 & 4981 & 121 & 5466 & 298 & 65.04 & 16.72 \\ \hline
20k(0.5-2) & 20 000    & 292526 & 3225 & 116 & 7935 & 284 & 36.87 & 11.36 \\ \hline
\end{tabular}
\end{table}

%\todo[inline]{v tabulke chybaju vysledky pre MLS a preparation time MCH na full grafe}
%\todo[inline]{ s full graf je problem. nahodne dozaci tam casto nezbehnou ani za nekolik hodin. jsou dve reseni but tam nedat to MLS pro porovnani a nebo dat vyber jen tech dotazu co zbehnout (mam uz ho pripraveny). Ktera varianta je lepsi?}

The {\em ``speed up expanded''} and {\em ``speed up time''} columns in the table express how many times faster the MCH queries were compared to MLS queries with respect to number of expanded nodes and computation time respectively. We can see that MCH performed better on all the benchmark instances with the exception of the smallest one and modified smallest one ({\tt 11k-10k} and {\tt 11k-10k-(0.75-1.5)}), where MLS had shorter computation time (the speedup was $0.57 < 1$, $0.2~<~1$).


% jak porametry MCq ovlivnovali dobu preprocessingu:
%\subsection{Graph preprocessing}
%čím dál hýbat:
%CHP - neupdatovat frontu pořád
%Řadit podle počtu zkratek / podle okoli / stupeň (DynamicNode)
%Čas ukoncění prohledávání okolo
%(nenaimplementovane: nekontrohovat cele)
%
%každé testovat na třech instancích
%pro každou uvést: 
%\#hran \#zkratek \% \%delky cesty \%zkraceni čas
%
%TODO pocet stratek a cas pri precessing celeho grafu
% weighted ordering?
%
\begin{table}
\caption{Preprocessing ...}
\label{table1}
\begin{tabular}{l|l|c|ccc|ccc}
    instance & variant & prep & edges & shortcuts & \% & MLS length & MCH length & \% \\ \hline
\multicolumn{9}{c}{Summary} \\ \hline
2455-20k & default & 27s & 49296 & 35081 & 1.71 & 134.84 & 20.98 & 6.43 \\ \hline
5555-20k & default & 21s & 50292 & 36137 & 1.71 & 152.41 & 18.40 & 8.28 \\ \hline 
8765-20k & default & 10s & 51090 & 34785 & 1.68 & 103.76 & 17.17 & 6.04 \\ \hline
2455-20k & \#s & 23s & 49296 & 35192 & 1.71 & 100.99 & 17.42 & 5.80 \\ \hline
5555-20k & \#s & 25s & 50292 & 37123 & 1.74 & 124.99 & 16.09 & 7.76 \\ \hline
8765-20k & \#s & 8s & 51090 & 34964 & 1.68 & 88.73 & 15.71 & 5.64 \\ \hline
2455-20k & \#s-d & 47s & 49296 & 53956 & 2.09 & 127.07 & 16.30 & 7.79 \\ \hline
5555-20k & \#s-d & 43s & 50292 & 56038 & 2.11 & 125.09 & 19.35 & 6.46 \\ \hline
8765-20k & \#s-d & 29s & 51090 & 57701 & 2.13 & 94.32 & 17.43 & 5.41 \\ \hline

%Average & variant a & & & & & \\ \hline
%Average & variant b & & & & & \\ \hline
%Average & variant c & & & & & \\ \hline
\end{tabular}
\end{table}

