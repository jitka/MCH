% 4. Experiments
%  - We've tried both algorithms (MCHp and MCHq) on the New York City road 
%    map benchmark from ... (what was that challenge? include the lint to it)
%  - The full graph had 270K nodes. Benchmark also contained subgraphs with 10,20,30-50K nodes. 
%    There were X queries.
%  - We compared our algorithms to alternatives A, B, C... (reference them - was it only Dijkstra?)
%  - (optional) Link to our implementation is ... 
%  - The experiments were conducted on a machine ... (specifications)
%  - The experiment results are in ... (results table here)
%  - In the table, we see that our algorithms performed better with respect to 
%    number of expanded nodes and with respect to query time (in all cases?).
%    On average, we expanded only X% of the nodes expanded by Dijkstra and our queries 
%    took only Y% of Dijkstra time.
% politika zabijeni whitnessearche
% heuristika


\section{Experiments}
In this section, we present the results of computations measuring the performance
of MCHp and MCHq versus MLS.

We've tried both algorithms (MCHp and MCHq) on the New York City road 
map benchmark from the 9th DIMACS Implementation
Challenge: Shortest Path 
\url{http://www.dis.uniroma1.it/challenge9/}

The full graph had 270K nodes. Benchmark also contained subgraphs with 10,20,30-50K nodes. 

The New York City benchmark contains two criteria -- distance and time. 
We generate graphs with random third criterion. 
When original weight vector is $(w_1,w_2)$ the new is 
$(w_1,w_2,rand(0.75,1.5) \times w_1 + rand(0.75,1.5) \times w_2)$ and
$(w_1,w_2,rand(0.5,2) \times w_1 + rand(0.5,2) \times w_2)$.
\todo{pro tyhle grafy mam vysledky spocitane v raw\_result.txt, do tabulky doplnim pozdej}

There were 1000 random queries on bi-criteria subgraphs and 10 queries
on full graph and tri-criteria graphs. 
\todo{ty dotazy na uplny graf jsem vybyrala tak aby MLS dobehlo, protoze na nektere dotazy bezi dele nez den, nevim jak to napsat}

Rank function for MCH is 
% return this.shortcut*3+deleted*3-inEdges-outEdges;
Witness search has time limit 0.25s. 

We compared our algorithms to MLS.
\todo{( was it only Dijkstra?), psala jsem i namou, ale vycházela mi jen asi o 30\% rychlejsi nez MLS, a nejsem si jista jestli je to ok a nebo mam spatnou imlementaci nebo spatnou heuristiku}

All algorithms were implemented in Java 1.8 
The experiments were conducted on a machine with 
32 GB or RAM and  Intel(R) Core(TM) i7-4770S CPU @ 3.10GHz processor.

The experiment results are in Table \ref{result}. 

In the table, we see that our algorithms performed better with respect to 
number of expanded nodes and with respect to query time.
On average, we expanded only X\% of the nodes expanded by Dijkstra and our queries 
 took only Y\% of Dijkstra time.


\begin{table}
\caption{Comparison of MLS and MCL}
\label{result}
\begin{tabular}{l|c|cc|ccccc}
 & & \multicolumn{2}{c}{DIJK} & \multicolumn{5}{c}{MCH} \\ \cline{3-6}
instance & nodes & query & query & prep & query & query & speed up & speed up \\ \hline
& & (expanded) & (ms) & (min) & (expanded) & (ms) & expanded & time \\ \hline
full & 270 000 & & & &  2 & 8 & 1.1 & 1.9 \\ \hline
testGraph11k-10k & 10 000 & 26051 & 32 & 4.3 & 4227 & 56 & 6.16 & 0.57 \\ \hline
testGraph2455-20k & 20 000 & 178472 & 607 & 31 & 2055 & 34 & 86.85 & 17.85 \\ \hline
testGraph5555-20k & 20 000 & 174808 & 514 & 23 & 2004 & 20 & 87.23 & 25.7 \\ \hline
testGraph8765-20k & 20 000 & 53238 & 60 & 12 & 1365 & 5 & 39.00 & 12.0 \\ \hline
testGraph110k-30k & 30 000 & 146709 & 278 & 41 & 6853 & 149 & 21.41 & 1.86 \\ \hline
testGraph150k-40k & 40 000 & 912326 & 17455 & 1978 & 10307 & 353 & 88.51& 49.45 \\ \hline
testGraph200k-50k & 50 000 & 996507 & 16181 & 2142 & 13681 & 711 & 72.84& 22.76 \\ \hline
    testGraph11k-10k-(0.75-1.5) & 10 000 & & & & & & & \\ \hline
    testGraph11k-10k-(0.5-2) & 10 000 & & & & & & & \\ \hline
    testGraph2455-20k-(0.75-1.5) & 20 000 & & & & & & & \\ \hline
    testGraph2455-20k-(0.5-2) & 20 000 & & & & & & & \\ \hline

\multicolumn{9}{c}{Summary} \\ \hline
Average & 27 143 & 26051 & 32 & 4.3 & 1365 & 5 & 6.16 & 88.51 \\ \hline
Minimum & 10 000 & 996507 & 17455 & 2142 & 13681 & 711 & 0.57 & 49.45\\ \hline
Maximum & 50 000 & 355444 & 5018 & 604 & 5786 & 190 & 57 & 18.6 \\ \hline
\end{tabular}
\end{table}

% jak porametry MCq ovlivnovali dobu preprocessingu:
%\subsection{Graph preprocessing}
%čím dál hýbat:
%CHP - neupdatovat frontu pořád
%Řadit podle počtu zkratek / podle okoli / stupeň (DynamicNode)
%Čas ukoncění prohledávání okolo
%(nenaimplementovane: nekontrohovat cele)
%
%každé testovat na třech instancích
%pro každou uvést: 
%\#hran \#zkratek \% \%delky cesty \%zkraceni čas
%
%TODO pocet stratek a cas pri precessing celeho grafu
% weighted ordering?
%
%\begin{table}
%\caption{Preprocessing ...}
%\label{table1}
%\begin{tabular}{l|l|ccc|cc}
%instance & variant & edges \# & shortcut \# & shortcut X & length \# & shorter x \\ \hline
%\multicolumn{7}{c}{Summary} \\ \hline
%Average & variant a & & & & & \\ \hline
%Average & variant b & & & & & \\ \hline
%Average & variant c & & & & & \\ \hline
%\end{tabular}
%\end{table}

