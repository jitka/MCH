% 4. Experiments
%  - We've tried both algorithms (MCHp and MCHq) on the New York City road 
%    map benchmark from ... (what was that challenge? include the lint to it)
%  - The full graph had 270K nodes. Benchmark also contained subgraphs with 10,20,30-50K nodes. 
%    There were X queries.
%  - We compared our algorithms to alternatives A, B, C... (reference them - was it only Dijkstra?)
%  - (optional) Link to our implementation is ... 
%  - The experiments were conducted on a machine ... (specifications)
%  - The experiment results are in ... (results table here)
%  - In the table, we see that our algorithms performed better with respect to 
%    number of expanded nodes and with respect to query time (in all cases?).
%    On average, we expanded only X% of the nodes expanded by Dijkstra and our queries 
%    took only Y% of Dijkstra time.
% politika zabijeni whitnessearche
% heuristika

% Rozdelit Experiment Setting, Experiments, idealne i Discusion


\section{Experiments}
\label{secExperiments}

\subsection{Experiment Setting}
We experimented with our implementation of MCHp and MCHq algorithms using the New York City road map benchmark from 9th DIMACS Implementation Challenge: Shortest Path\footnote{
\url{http://www.dis.uniroma1.it/challenge9/}} and compared them to MLS algorithm. We were interested in two metrics: number of expanded nodes and computation time.

Graph representing New York City road network consists of 264346 nodes and 733846 edges. In addition to full graph, the benchmark also features smaller connected subgraphs with 10K, 20K, 30K, 40K and 50K nodes. 

The benchmark problem was defined as a bi-criterial shortest path search. The two objectives were: geographic distance and travel time. 

We also generated additional graphs with a randomized third criterion in order to analyze how the algorithm depends on the correlation of weights. Where the original weight vector was $(w_1,w_2)$, we set the new vector to be $(w_1,w_2,rand(0.75,1.5) \times w_1 + rand(0.75,1.5) \times w_2)$ and $(w_1,w_2,rand(0.5,2) \times w_1 + rand(0.5,2) \times w_2)$. 

%We compared our algorithms to MLS.
%\todo{( was it only Dijkstra?), psala jsem i namou, ale vycházela mi jen asi o 30\% rychlejsi nez MLS, a nejsem si jista jestli je to ok a nebo mam spatnou imlementaci nebo spatnou heuristiku}

All the algorithms were implemented in Java 1.8. The experiments were conducted on a machine with 
32GB RAM and Intel(R) Core(TM) i7-4770S CPU @ 3.10GHz processor.


\subsection{Query Phase Experiment Results}

We computed 1,000 random queries on the subgraphs and 10 queries on the full graph and tri-criteria graphs and averaged the results (times and numbers of extended nodes) among those queries. 
On the full graph, we only considered those queries, where MLS algorithm finished in a reasonable time. The queries where MLS failed to finish within one day were not included in the results. 

The {\em node rank} heuristic was computed as {\tt shortcut*3 + deleted*3 -~inEdges -~outEdges} in our MCH implementation. The Witness Search time limit was set to 0.25 sec. 

\noindent The experiment results are in Table \ref{result}. 

\begin{table}[h]
\tiny
\caption{Comparison of MLS and MCH}
\label{result}
\begin{tabular}{l|c|cc|ccccc}
 & & \multicolumn{2}{c}{MLS} & \multicolumn{5}{c}{MCH} \\ \cline{3-6} \hline
instance & nodes & query & query & prep & query & query & speed up & speed up \\ \hline
& & (expanded) & (ms) & (sec) & (expanded) & (ms) & expanded & time \\ \hline
full & 270 000 & 9883105 & 494352 & $\sim10^5$ & 36709 & 9289 & 269.23 & 53.22  \\ \hline
11k-10k & 10 000 & 26051 & 32 & 4.3 & 4227 & 56 & 6.16 & 0.57 \\ \hline
2455-20k & 20 000 & 178472 & 607 & 31 & 2055 & 34 & 86.85 & 17.85 \\ \hline
5555-20k & 20 000 & 174808 & 514 & 23 & 2004 & 20 & 87.23 & 25.7 \\ \hline
8765-20k & 20 000 & 53238 & 60 & 12 & 1365 & 5 & 39.00 & 12.0 \\ \hline
110k-30k & 30 000 & 146709 & 278 & 41 & 6853 & 149 & 21.41 & 1.86 \\ \hline
150k-40k & 40 000 & 912326 & 17455 & 1978 & 10307 & 353 & 88.51& 49.45 \\ \hline
200k-50k & 50 000 & 996507 & 16181 & 2142 & 13681 & 711 & 72.84& 22.76 \\ \hline
10k(0.75-1.5) & 10 000  & 38997 & 126 & 17 & 8830 & 643 & 4.42 & 0.20 \\ \hline
10k(0.5-2) & 10 000     & 377541 & 20798 & 734 & 12261 & 353 & 30.80 & 58.92 \\ \hline
20k(0.75-1.5) & 20 000 & 355497 & 4981 & 121 & 5466 & 298 & 65.04 & 16.72 \\ \hline
20k(0.5-2) & 20 000    & 292526 & 3225 & 116 & 7935 & 284 & 36.87 & 11.36 \\ \hline
\end{tabular}
\end{table}

The {\em ``speed up expanded''} and {\em ``speed up time''} columns in the table express how many times faster the MCH queries were compared to MLS queries with respect to number of expanded nodes and computation time respectively. We can see that MCH performed better on all the benchmark instances with the exception of the smallest one and modified smallest one ({\tt 11k-10k} and {\tt 11k-10k-(0.75-1.5)}), where MLS had shorter computation time (the speedup was $0.57 < 1$, $0.2~<~1$).


\subsection{Preprocessing Phase Experiment Results}

It is possible to use different {\em node rank} heuristics during the
preprocessing phase. We experimented with the following three variants 
and compared them in table \ref{result2}:

\begin{itemize}
    \item {\bf default rank heuristic:} {\tt shortcut*3 + deleted*3 -~inEdges -~outEdges} 
    \item {\bf no. of shortcuts (\#s):} {\tt shortcut} 
    \item {\bf shortcuts - degree (\#s-d):} {\tt shortcut -~inEdges -~outEdges} 
\end{itemize}

We tested these three variants of rank heuristics on three 
subgraphs and computed the average values.

The preprocessing phase was fastest with the {\em \#s} heuristic (no. of shortcuts), as seen in the {\em prep} column of Table \ref{result2}. 
It is interesting that the {\em \#s-d} heuristic (shortuct - degree), which should greedily 
minimize the number of new shortcuts created considerably more shortcuts than the others (see the {\em shortcuts} column).

The most important parameter we analyzed was the average length (number of nodes) of 
the paths in the Pareto set before reconstructing the paths to the original graph, 
since shorter paths potentially decrease the computation time of the Query phase algorithm (MCHq).
For each test, we generated 100 random queries and computed the average number of nodes 
in all the found paths. We computed this number for both the MLS and MCH answers. As you can see 
in the {\em MCH length} column, the {\em \#s} heuristic (no. of shortcuts) appears to 
be the best in this respect, with the average length of 16.41 nodes.  

% jak porametry MCq ovlivnovali dobu preprocessingu:
%\subsection{Graph preprocessing}
%čím dál hýbat:
%CHP - neupdatovat frontu pořád
%Řadit podle počtu zkratek / podle okoli / stupeň (DynamicNode)
%Čas ukoncění prohledávání okolo
%(nenaimplementovane: nekontrohovat cele)
%The {\em node rank} heuristic was computed as {\tt shortcut*3 + deleted*3 -~inEdges -~outEdges} in our MCH implementation. The Witness Search time limit was set to 0.25 sec. 
%
\begin{table}[h]
\tiny
\caption{Preprocessing of MCH}
\label{result2}
\begin{tabular}{l|l|c|ccc|ccc}
    instance & variant & prep & edges & shortcuts & \% & MLS length & MCH length & \% \\ \hline
2455-20k & default & 27s & 49296 & 35081 & 1.71 & 134.84 & 20.98 & 6.43 \\ \hline
5555-20k & default & 21s & 50292 & 36137 & 1.71 & 152.41 & 18.40 & 8.28 \\ \hline 
8765-20k & default & 10s & 51090 & 34785 & 1.68 & 103.76 & 17.17 & 6.04 \\ \hline
2455-20k & \#s & 23s & 49296 & 35192 & 1.71 & 100.99 & 17.42 & 5.80 \\ \hline
5555-20k & \#s & 25s & 50292 & 37123 & 1.74 & 124.99 & 16.09 & 7.76 \\ \hline
8765-20k & \#s & 8s & 51090 & 34964 & 1.68 & 88.73 & 15.71 & 5.64 \\ \hline
2455-20k & \#s-d & 47s & 49296 & 53956 & 2.09 & 127.07 & 16.30 & 7.79 \\ \hline
5555-20k & \#s-d & 43s & 50292 & 56038 & 2.11 & 125.09 & 19.35 & 6.46 \\ \hline
8765-20k & \#s-d & 29s & 51090 & 57701 & 2.13 & 94.32 & 17.43 & 5.41 \\ \hline
\multicolumn{9}{c}{Summary} \\ \hline
Average & default & 19.33 & 50226.0 & 35334.33 & 1.7 & 130.33 & 18.85 & 6.92 \\ \hline
Average & \#s & 18.67 & 50226.0 & 35759.67 & 1.71 & 104.90 & 16.41 & 6.40 \\ \hline
Average & \#s-d & 39.67 & 50226.0 & 55898.33 & 2.11 & 115.49 & 17.69 & 6.55 \\ \hline
\end{tabular}
\end{table}

%[[27,49296,35081,1.71,134.84,20.98,6.43], [21,50292,36137,1.71,152.41,18.40,8.28], [10,51090,34785,1.68,103.76,17.17,6.04], [23,49296,35192,1.71,100.99,17.42,5.80], [25,50292,37123,1.74,124.99,16.09,7.76], [8,51090,34964,1.68,88.73,15.71,5.64], [47,49296,53956,2.09,127.07,16.30,7.79], [43,50292,56038,2.11,125.09,19.35,6.46], [29,51090,57701,2.13,94.32,17.43,5.41]]

