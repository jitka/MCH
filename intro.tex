% 1. Introduction:
% 
%  (background)
%  - Common problem: Multi-objective shortest path problems (MOSP)
%  - Briefly explain the problem in natural language (no formalisms)
%  - How this graph problem relates to real world (path planning)
%  - The easier variant of the problem, SPP, can be solved 
%    using Contraction Hierarchies (CH)
%  - In 1-2 natural language sentences, explain the contraction hierarchies approach
%  
%  (problem)
%  - CH has good results (reference), but it was no used for MOSP yet
%  
%  (solution)
%  - in this paper, we show how CH can be used for MOSP
%  - [describe the solution] 
%  
%  (outline)
%  - paper outline. contents of all the sections

\section{Introduction}

Contraction hierarchies (CH) \cite{geisberger2008contraction}
is fundamental improvement for single criteria path graphs.

TODO update 
Chapter two present generalization for highway dimension. In chapter
three we present complexity bound for multi-criteria contraction hierarchies.
At chapter four we present our results. First is number of transition roads for
New York street graph which support our belief that multi-criteria pathway
has also low highway dimension.
Then we present comparison with other algorithms.


% X. Related Work
%  - Delete this section and have all the references in section 1. Introduction.

Constrained shortest path 
\cite{pugliese2013survey} is often used in cases when we are looking for
paths which are good in more than one criteria.
CH could be used for finding 
paths which are on lover envelope of paretoset.
\cite{funke2013polynomial}
or for the other constrains \todo{cite electromobility}
However, minimalising linear combination of criteria sum 
do not found all paths from paretoset.

We present algorithm which found whole paretoset.
Multicriterial dijkstra is optimal for general graphs.
\todo{citace} 
For vehicle or bicycle routing could be used namoa
\todo{citace}
which is optimal for direct questions.
There could be improvemenents when we allows preprocessing
algorithms 
There was preprocessing algorithms for multi-criteria
shortest path already.
\cite{kohler2005acceleration} 

Higway dimension  \cite{abraham2010highway} is parametr which 
is low on road network whit only on criteria
and it could be used for upper bound of time complexity of search 
algorithms. We generalize highway dimension in orther to found 
complexity bound also for our problem.

%Intuice v pozadí:
%Important vertices plati nejen čase - dokazano v
%ale věřím že bude funguvat i v namořské výšce a
%pohodlnosti cest
%Tj delší důležitější cesty mají hezké vlastnosti

%tady patri vetsina uvodu

Fully dynamic speed-up techniques for multi-criteria shortest path searches in time-dependent networks.
\cite{berger2010fully}

\todo{cite bidirectional heuristic}
\cite{holte2016bidirectional}
\todo{future work hub labeling}
\cite{abraham2012hierarchical}




%%%%%%%%%%%%%%%%%%%%%%%%%%%%%%%%%%%%%%%%%%%%%%%%%%%%%%%%%%%%%%%%%%%%%%%%%%%%%%%%

