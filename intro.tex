% 1. Introduction:
% 
%  (background) - Common problem: Multi-objective shortest path problems (MOSP)
%  - Briefly explain the problem in natural language (no formalisms) - How this
%  graph problem relates to real world (path planning) - The easier variant of
%  the problem, SPP, can be solved using Contraction Hierarchies (CH) - In 1-2
%  natural language sentences, explain the contraction hierarchies approach
%  
%  (problem) - CH has good results (reference), but it was no used for MOSP yet
%  
%  (solution) - in this paper, we show how CH can be used for MOSP - [describe
%  the solution] 
%  
%  (outline) - paper outline. contents of all the sections

\section{Introduction} \label{secIntroduction}

Multi-objective shortest path problem (MOSP) is a fundamental problem from
graph theory with many real world applications. Algorithms solving the problem
need to find the Pareto set of shortest paths between two nodes in a
multi-weighted directed or undirected graphs or multi-graphs. 

A typical example of real-world application of MOSP is automated planning of
paths within a transport network represented as a graph, while optimizing
criteria such as travel time, distance, price, comfort or compatibility with
traveller's preferences. This problem arises from both travel industry and
daily life activities. Trip planning applications, such as car navigation
systems need efficient algorithms for for multi-criteria problems in very large
transport networks~\cite{veneti2016time}. 

% smazala jsem public transport protoze to je time-dependent search coz je dost
% jiny problem

%  (problem) - CH has good results (reference), but it was no used for MOSP yet
%  CSH

One of the common methods used to speed up the shortest path search  is called
Contraction Hierarchies (CH)~\cite{geisberger2008contraction}. Contraction
Hierarchies can be used to find a shortest path more efficiently than previous
routing approaches and is used in many advanced routing
techniques~\cite{delling2009engineering}.  CH is relatively simple algorithm
which could be modified so we choose it for generalisation above current state
of the art algorithm.

The problem is that Contraction Hiearchies is not applicable to multi-weighted
graphs and therefore cannot be used for MOSP, even though the method was
recently used for a similar problem -- Constrained Shortest Path Problem
(CSPP)~\cite{pugliese2013survey,funke2013polynomial}. Similarly to
Multi-objective shortest path, Constrained Shortest Path algorithms also search
for shortest paths in multi-weighted graphs, but instead of complete Pareto
sets, they only return a single path with the best linear combination of
weights.  

%  (solution) - in this paper, we show how CH can be used for MOSP - [describe
%  the solution] 

In this paper, we present an algorithm that generalizes the idea behind Contraction
Hierarchies approach to efficiently solve MOSP. Our algorithm, called
Multi-weighted Contraction Hierarchies (MCH), is able to find the whole Pareto
set of shortest paths between a pair of nodes in a multi-weighted directed
multi-graph. In addition to CH, we also draw inspiration from Multicriterial
Label-Setting (MLS) algorithm~\cite{martins1984multicriteria}. 
% \todo{citace} co si rovnou nezapisu na posledni chvili nenajdu!

%For vehicle or bicycle routing could be used namoa \todo{citace} which is
%optimal for direct (no preparation) questions.  \todo[inline]{to je state of
%the art a u te si nejsem jista jestli jsem si naimplementovala dobre a tak ji
%nemam ve vysledcich, nevim jestli ji davat tady.} -> radsej nie. 

%There could be improvemenents when we allows preprocessing algorithms. 

% Higway dimension  \cite{abraham2010highway} is parametr which is low on road
% network whit only on criteria and it could be used for upper bound of time
% complexity of search algorithms. We generalize highway dimension in orther to
% found complexity bound also for our problem.  
%Intuice v pozadí: Important vertices plati nejen čase - dokazano v ale věřím
%že bude funguvat i v namořské výšce a pohodlnosti cest Tj delší důležitější
%cesty mají hezké vlastnosti

%tady patri vetsina uvodu

%Fully dynamic speed-up techniques for multi-criteria shortest path searches in
%time-dependent networks.\cite{berger2010fully}

% bidirectional heuristic (spis nepouzitelna) \cite{holte2016bidirectional}
% future work hub labeling \cite{abraham2012hierarchical}


%  (outline) - paper outline. contents of all the sections
After the introduction of required terminology in Section
\ref{secPreliminaries}, we describe our two-stage solution in Section
\ref{secAlgorithms} -- MHC consists of two algorithms: one for 
graph preprocessing and one for queries. In Section
\ref{secExperiments}, we describe our experiments on DIMACS New York City road
network benchmarks and compare the performance to an alternative approach.
Finally, in Section \ref{secFormalResults}, we formally prove our algorithm's
correctness.

%%%%%%%%%%%%%%%%%%%%%%%%%%%%%%%%%%%%%%%%%%%%%%%%%%%%%%%%%%%%%%%%%%%%%%%%%%%%%%%%
\ptodo[inline]{Tato sekcia potrebuje podstatne viac referencii.}
