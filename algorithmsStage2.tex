%  3.2. Stage 2: Search query
%    - Explain MCHq in natural language here.
%      Get inspiration from what you wrote on whiteboard yesterday. 
%      Feel free to use some numbered lists to make it easy to read and understand.
%      If needed, make and use some graph pictures.
%    - Show the MCHq pseudocode.

\subsection{Stage 2: Search Query}
\label{subsecStage2}

Multi-weighted Contraction Hierarchies query (MCHq) is based on a bi-directional search.
Each $s$-$t$ search query runs two MLS-like searches on a slightly modified preprocessed graph
and then they combine their results.

Both forward and reverse search have structure open\_labels implemented as priority queue which orders labels alphabetically. MCHq searches on slightly modified graphs. 
We do not transform the whole graph in advance but modify it during queries.
\todo[inline]{Nie su v tomto odstavci nahodou len implementacne detaily, ktore pre samotny algoritmus nie su dolezite? Neda sa to vynechat?}

\paragraph*{Forward Search}
Forward search uses a preprocessed graph with edges oriented from the node with lower rank to nodes with higher rank. This could be done by replacing line \todo{budu predelavat sazbu pseudokodu aby to slo} by \ref{}
It runs MLS search algorithm with the query from node $s$ to $t$ and returns all label bug.\todo{label bug? co to je?}
Line TODO could by replaced by:
\todo[inline]{Priznam sa, ze vobec nerozumiem tomuto odstavcu (Forward Search). Chce to nejak preformulovat. Mimochodom, preco chceme rozpravat o nejakom nahradzani riadkov pseudokodu? }

\begin{lstlisting}[caption={Forward Search},label=list:8-6,captionpos=t,float,abovecaptionskip=-\medskipamount]
    for every outgoing edge e=(node(l),w):
    	if rank(node(l)) < rank(w):
        	continue;
\end{lstlisting}

\paragraph*{Reverse Search}
Reverse search uses graph with all the edges reversed and then keeps only edges
from lower rank to higher. 
This could by done by replacing line \todo{budu predelavat sazbu pseudokodu aby to slo} by \ref{}
It run MLS search algorithm with the query from node $t$ to $s$ and return all label bug.

\todo[inline]{Tak isto ako pri Forward Search - toto treba prepisat. Okrem toho - ma zmysel mat 2 odstavce a 2 pseudokody na Forward/Backward search, ked je to skoro to iste?}

\begin{lstlisting}[caption={Reverse Search},label=list:8-6,captionpos=t,float,abovecaptionskip=-\medskipamount]
    for every outgoing edge e=(w, node(l)):
    	if rank(w) < rank(node(l)):
        	continue;
\end{lstlisting}

To find the paths in preprocessed graph, we combine the labels from forward and reverse search. 
For all the nodes $v$ with $B(v) \ne \emptyset$ we join all the combinations of $s$-$v$ and $v$-$t$ paths. All non-dominated paths determine the Pareto set of shortest $s$-$t$ paths in the preprocessed graph.
Pareto set of shortest paths in the original graph can be reconstructed recursively using the notes on the shortcuts.

