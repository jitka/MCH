%  3.2. Stage 2: Search query
%    - Explain MCHq in natural language here.
%      Get inspiration from what you wrote on whiteboard yesterday. 
%      Feel free to use some numbered lists to make it easy to read and understand.
%      If needed, make and use some graph pictures.
%    - Show the MCHq pseudocode.

\subsection{Stage 2: Search Query}
\label{subsecStage2}

Multi-weighted Contraction Hierarchies query (MCHq) is based on a bi-directional search.
Each $s$-$t$ search query runs two MLS-like searches on a slightly modified preprocessed graph
and then we combine their results. 
%We do not transform the whole graph in advance -- we merely modify it during queries.

\paragraph*{Forward Search}
Forward search runs on a preprocessed sub-graph with only those oriented edges going from nodes with lower rank to nodes with higher rank (other edges are ignored). It is not necessary to transform the whole graph in advance -- we do it on the go. 
%In the pseudocode of MLS we change line \todo{jina sazba pseudokodu} to \ref{}.
The MLS algorithm is used to search for the shortest path from $s$ to $t$ and obtain the label bags $B(v)$ for all the nodes.

\begin{lstlisting}[caption={Line X of MLS for Forward Search},label=list:8-6,captionpos=t,float,abovecaptionskip=-\medskipamount]
    for every outgoing edge e=(node(l),w):
    	if rank(node(l)) < rank(w):
        	continue;
\end{lstlisting}

\paragraph*{Reverse Search}
Reverse search is similar to Forward search, but it originates in the node $t$ and runs on a graph with all the edges reversed (implicitly -- we do not need to actually modify the graph structure for this). Again, we only consider those edges, which (after being reversed) go from lower-ranked nodes to higher-ranked nodes. 
In this case, the MLS search algorithm is used to search for the shortest path from $t$ to $s$ and obtain the label bags $B(v)$ for all the nodes.

\vskip 5mm

\begin{lstlisting}[caption={Line X of MLS for Reverse Search},label=list:8-6,captionpos=t,float,abovecaptionskip=-\medskipamount]
    for every outgoing edge e=(w, node(l)):
    	if rank(w) < rank(node(l)):
        	continue;
\end{lstlisting}

To find the paths in preprocessed graph, we combine the labels from forward and reverse search. 
For all the nodes $v$ with $B(v) \ne \emptyset$, we join all the combinations of $s$-$v$ and $v$-$t$ paths. 
All non-dominated paths make up the Pareto set of shortest $s$-$t$ paths in the preprocessed graph.
Pareto set of shortest paths in the original graph can be reconstructed recursively using the notes on the shortcuts.

