%  3.2. Stage 2: Search query
%    - Explain MCHq in natural language here.
%      Get inspiration from what you wrote on whiteboard yesterday. 
%      Feel free to use some numbered lists to make it easy to read and understand.
%      If needed, make and use some graph pictures.
%    - Show the MCHq pseudocode.

\subsection{Stage 2: Search Query}
\label{subsecStage2}

Multi-weighted Contraction Hierarchies query (see Pseudocode \ref{MCHq} below) is based on bi-directional search.
Each $s$-$t$ search query runs two MLS-like searches on a slightly modified preprocessed graph
and then we combine their results. 
%We do not transform the whole graph in advance -- we merely modify it during queries.

\paragraph*{Forward Search}
Forward search runs on a preprocessed sub-graph with only those oriented edges going from 
lower-rank nodes to higher-rang nodes
(other edges are ignored). It is not necessary to transform the whole graph in advance -- we do it on the go. 
In the pseudocode of MLS we change line to \ref{MLS:edge} to code \ref{MCHqf}.
The MLS algorithm is used to search for the shortest path from $s$ to $t$ and obtain the label bags $B(v)$ for all the nodes.

\begin{algorithm}[H]
    \SetAlgoLined
    \SetKw{Continue}{continue}
    \caption{Line \ref{MLS:edge} of MLS for Forward Search}
    \label{MCHqf}
    \ForEach{ \upshape outgoing edge $e=(node(l),w)_i$}{
        \If { rank(node(l)) < rank(w)) } {
            \Continue
        }
        $\dots$
    }
\end{algorithm}

\paragraph*{Reverse Search}
Reverse search is similar to Forward search, but it originates in the node $t$ and runs on a graph with all the edges reversed (implicitly -- we do not need to actually modify the graph structure for this). Again, we only consider those edges, which (after being reversed) go from lower-ranked nodes to higher-ranked nodes. 
In the pseudocode of MLS we change line to \ref{MLS:edge} to code \ref{MCHqr}.
In this case, the MLS search algorithm is used to search for the shortest path from $t$ to $s$ and obtain the label bags $B(v)$ for all the nodes.

\vskip 5mm

\begin{algorithm}[H]
    \SetAlgoLined
    \SetKw{Continue}{continue}
    \caption{Line \ref{MLS:edge} of MLS for Reverse Search}
    \label{MCHqr}
    \ForEach{ \upshape outgoing edge $e=(w,node(l))_i$}{
        \If { rank(w) < rank(node(l)) } {
            \Continue
        }
        $\dots$
    }
\end{algorithm}

To find the paths in preprocessed graph, we combine the labels from forward and reverse search. 
For all the nodes $v$ with $B(v) \ne \emptyset$, we join all the combinations of $s$-$v$ and $v$-$t$ paths. 
All non-dominated paths make up the Pareto set of shortest $s$-$t$ paths in the preprocessed graph.
Pareto set of shortest paths in the original graph can be reconstructed recursively using the notes on the shortcuts.

\begin{algorithm}[H]
    \SetAlgoLined
    \SetKw{Continue}{continue}
    \SetKwInOut{Input}{input}
    \SetKwInOut{Output}{output}
    \caption{Algorithm MCHq}
    \label{MCHq}
    \Input{ Preprocessed graph $G'$, source node $s$, target node $t$.}
    forward\_bags $\leftarrow$ forward search \\
    backward\_bags $\leftarrow$ backward search \\
    pareto\_set $\leftarrow \emptyset$ \\
    \ForEach{ \upshape node $v \in $ forward\_bags } {
        $L \leftarrow $ forward\_bags($v$) \\
        $R \leftarrow $ reverse\_bags($v$) \\
        \ForEach{ \upshape par of label $l \in L, r \in R$ } {
            \ForEach{ \upshape label $k$ from pareto\_set}{
                \If{ \upshape $w(k)$ is dominated by $w(l)+w(r)$ } {
                    remove $k$ from pareto\_set
                }
            }
            \If{ \upshape $l$ is not dominated by any label from pareto\_set } {
                pareto\_set $\leftarrow$ path from $l$ connected to reverse path from $r$
            }
        }
    }
    \ForEach{ \upshape path $p$ $\in$ pareto\_set } {
        reconstruct path to original graph
    }
    \Output{ Pareto set of shortest $s$-$t$ paths. }
\end{algorithm}
