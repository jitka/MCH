%  3.2. Stage 2: Search query
%    - Explain MCHq in natural language here.
%      Get inspiration from what you wrote on whiteboard yesterday. 
%      Feel free to use some numbered lists to make it easy to read and understand.
%      If needed, make and use some graph pictures.
%    - Show the MCHq pseudocode.

\subsection{Stage 2: Search Query}
\label{subsecStage2}

Multi-weighted Contraction Hierarchies query (MCHq) is based on a bi-directional search.
Each $s$-$t$ search query runs two MLS-like searches on a slightly modified preprocessed graph
and then they combine their results.

We do not transform the whole graph in advance but modify it during queries.

\paragraph*{Forward Search}
Forward search uses a preprocessed graph with edges oriented from the node with lower rank to nodes with higher rank. 
It is not necessary transform the whole graph in advance. 
The graph could be modify during queries.
In the pseudocode of MLS we change line \todo{jina sazba pseudokodu} to \ref{}.
We run MLS search algorithm with the query from node $s$ to $t$ and returns label bags $B(v)$ for all nodes.

\begin{lstlisting}[caption={Forward Search},label=list:8-6,captionpos=t,float,abovecaptionskip=-\medskipamount]
    for every outgoing edge e=(node(l),w):
    	if rank(node(l)) < rank(w):
        	continue;
\end{lstlisting}

\paragraph*{Reverse Search}
\todo[inline]{prepisu po schvaleni Forward Searche}
Reverse search uses graph with all the edges reversed and then keeps only edges
from lower rank to higher. 
This could by done by replacing line \todo{budu predelavat sazbu pseudokodu aby to slo} by \ref{}
It run MLS search algorithm with the query from node $t$ to $s$ and return all label bug.

\todo[inline]{Tak isto ako pri Forward Search - toto treba prepisat. Okrem toho - ma zmysel mat 2 odstavce a 2 pseudokody na Forward/Backward search, ked je to skoro to iste?}

\begin{lstlisting}[caption={Reverse Search},label=list:8-6,captionpos=t,float,abovecaptionskip=-\medskipamount]
    for every outgoing edge e=(w, node(l)):
    	if rank(w) < rank(node(l)):
        	continue;
\end{lstlisting}

To find the paths in preprocessed graph, we combine the labels from forward and reverse search. 
For all the nodes $v$ with $B(v) \ne \emptyset$ we join all the combinations of $s$-$v$ and $v$-$t$ paths. 
All non-dominated paths determine the Pareto set of shortest $s$-$t$ paths in the preprocessed graph.
Pareto set of shortest paths in the original graph can be reconstructed recursively using the notes on the shortcuts.

