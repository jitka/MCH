%  3.2. Stage 2: Search query
%    - Explain MCHq in natural language here.
%      Get inspiration from what you wrote on whiteboard yesterday. 
%      Feel free to use some numbered lists to make it easy to read and understand.
%      If needed, make and use some graph pictures.
%    - Show the MCHq pseudocode.

\subsection{Stage 2: Search query}

Each $s$-$t$ query is based on bi-directional search.
Both searches are based on Meta search algorithm with priority queue which orders labels
alphabetically. It searches on slightly modified graphs. We do not transform the whole
graph in advance but modify it during queries.

\emph{Forward search} use graph with edges
oriented from the node with lower rank to nodes with higher rank.
We run Meta Search algorithm with the query from node $s$ to $t$ and use all label bug.
Line TODO could by replaced by:

\begin{lstlisting}[caption={MCHp},label=list:8-6,captionpos=t,float,abovecaptionskip=-\medskipamount]
    for every outgoing edge e=(node(l),w):
    	if rank(node(l)) < rank(w):
        	continue;
\end{lstlisting}

\emph{Reverse search} use graph whit all edges reversed and then keep only edges
from lower rank to higher. 
The reverse search runs Meta Search algorithm with the query from node $t$ to $s$ and uses all label bug.
Line TODO could by replaced by:

\begin{lstlisting}[caption={MCHp},label=list:8-6,captionpos=t,float,abovecaptionskip=-\medskipamount]
    for every outgoing edge e=(w, node(l)):
    	if rank(w) < rank(node(l)):
        	continue;
\end{lstlisting}

To find paths in preprocessed graph is combine labels from forward and reverse search. 
For all nodes $v$ with $B(v) \ne \emptyset$ we join all combination of $s$-$v$ and $v$-$t$ paths. 
All non-dominated paths determine Pareto set of shortest $s$-$t$ path in preprocessed set.

Original paths could be reconstructed recursively.
Each shortcut $(u,v)_i$ could be decomposed either by using note or 
by comparing all out-edges of $u$ and all in-edges of $v$.



% TODO
%vic popsat uvod - nazvy
%nazvat veci
%Rozumne dělení, nadpisy
%PSEUDOKOD
%mohou vztikat multihrany
% updatovani